\documentclass[12pt,a4paper,english]{article}
\usepackage[utf8]{inputenc}
\usepackage[english]{babel}
\usepackage{graphicx}
\usepackage{epsfig}            % To allow figures
\usepackage{pstricks}          % To draw color pictures directly
\usepackage{fancyhdr}
\usepackage[verbose,a4paper,tmargin=30mm,bmargin=37.5mm,lmargin=30mm,rmargin=30mm]{geometry}
\usepackage{helvet}
\usepackage{mathptmx}
\usepackage[T1]{fontenc}
\usepackage{rotating}
\usepackage{amssymb}       % More special characters          
\usepackage{amsmath}       % More mathematical characters
\usepackage{natbib} % Needed e.g. for \citep
\usepackage{multirow}
\usepackage{verbatim}          % Use the verbatim styles
\definecolor{METblue}{cmyk}{0.85,0,0.2,0.2}
\usepackage{titlesec}
\titleformat{\section}
  [block]
  { \normalfont\sffamily\Large\bfseries\color{METblue} }
  {\makebox[2em][r]{\thesection}}
  {5mm}
  {\vspace{5mm}}

\titleformat{\subsection}[block]%
  {\normalfont\sffamily\bfseries}%
  {\makebox[2em][r]{\thesubsection}}%
  {5mm}
  {\vspace{3mm}}[]
\titleformat{\subsubsection}[block]%
  {\normalfont\sffamily\bfseries}%
  {\makebox[3em][r]{\thesubsubsection}}%
  {5mm}
  {\vspace{3mm}}[]

% Titlespacing syntax: 
\titlespacing*{\section}{-17mm}{5mm}{0mm}
\titlespacing*{\subsection}{-13.5mm}{5mm}{0mm}
\titlespacing*{\subsubsection}{-17.5mm}{5mm}{0mm}

%%% Formatting the table of contents
\usepackage{tocloft}
\renewcommand{\cfttoctitlefont}{\sffamily\bfseries\color{METblue}\Large}
\providecommand{\cftchapfont}{\sffamily\bfseries }
\renewcommand{\cftsecfont}{\sffamily\bfseries }
\renewcommand{\cftsubsecfont}{\sffamily }
\renewcommand{\cftsubsubsecfont}{\sffamily }
\providecommand{\cftsubsubsubsecfont}{\sffamily }
\renewcommand{\cftfigfont}{Figure }
\renewcommand{\cfttabfont}{Table }
\providecommand{\cftchappagefont}{\sffamily}
\renewcommand{\cftsecpagefont}{\sffamily\bfseries}
\renewcommand{\cftsubsecpagefont}{\sffamily}
\renewcommand{\cftsubsubsecpagefont}{\sffamily}
\providecommand{\cftsubsubsubsecpagefont}{\sffamily}

\renewcommand{\baselinestretch}{1.33}

% --------------------------------------

\begin{document}

\bibliographystyle{ams}
\thispagestyle{empty}  % Hide page numbers

\noindent
\begin{tabular}{@{} p{63mm} p{50mm} r}
\includegraphics*[]{met_rapport_logo_eng} % Automatically uses PDF or EPS in same directory depending on latex or pdflatex.
&
\fontsize{27.5pt}{33pt} \selectfont \bf \sffamily MET{\color{gray} report}
&
 \begin{minipage}[b]{28mm}
  \begin{flushright}
   \footnotesize \sffamily No. X/2015 \\ ISSN 2387-4201 \\ Oceanography              % Report number and Category
  \end{flushright}
 \end{minipage}
\end{tabular}

\vfill

\begin{flushright}
{\fontsize{36pt}{43.2pt}\selectfont \bf \sffamily A cruise in the Oslofjord}          % Title
{\fontsize{14.0pt}{16.8pt}\selectfont \bf \sffamily FjordOs technical report No. 3}   % Subtitle

\vspace{5mm}
{\fontsize{12.5pt}{15pt}\selectfont \sffamily January 2016                                          % Subtitle
\\
\sffamily Karina Hjelmervik, Nils Melsom Kristensen and Andre Staalstr\o m           % Author name(s)
}
\end{flushright}

%\vspace{25mm}
\vspace{2mm}

\begin{figure}[!h]
\begin{center}
\includegraphics*{met_rapport_monster}          % Graphic. Automatically uses PDF or EPS in same directory depending on latex or pdflatex.
\end{center}
\end{figure}

%\newpage

%\thispagestyle{empty}  % Hide page numbers

\clearpage

\setlength{\unitlength}{1mm}  %Needed for picture environment

\begin{table}[!ht]

\begin{tabular}[c]{lr}
\vspace{5mm}
\includegraphics*{met_rapport_logo_eng} & \hspace{43mm}
{\fontsize{27.5pt}{33pt}\selectfont \bf \sffamily MET{\color{gray} report}}\\
\end{tabular}

\sffamily{
\begin{tabular}[t]{|p{110mm}|p{40mm}|} \hline
{\bf \sffamily Title}                  & {\bf \sffamily Date}               \\ 
Cruise in the Oslofjord
                             & \today                   \\ \hline
{\bf \sffamily Section}                & {\bf \sffamily Report no.}         \\ 
 Type                        &  X/2016                  \\ \hline
{\bf \sffamily Author(s)}                 & {\bf \sffamily Classification}     \\ 
Karina Hjelmervik, Nils Melsom Kristensen and Andre Staalstr\o m
                             & \begin{picture}(20,4)(-2,-1.0)
                               \put (0,0){\circle*{4}}
                               \put (7,0){\makebox(0,0){Free}}
                               \put (15,0){\circle{4}}
                               \put (27,0){\makebox(0,0){Restricted}}
                               \end{picture}
                               \\ \hline
{\bf \sffamily Client(s)}              & {\bf \sffamily Client's reference} \\ 
Client name                  &               \\ \hline
\end{tabular}

\begin{tabular}[t]{|p{154.3mm}|}
{\bf \sffamily Abstract}                                          \\
A field trip onboard the research vessel R/V Trygve Braarud in the Oslofjord 21 – 23 September 2015 provided observations on hydrography and trajectories of drifters. The field trip was carried out as a part of the FjordOs project as a pilot test to examinie if more field test would benefit the validation of the new Oslofjord model, the FjordOs CL model.

The cruise was carried out shortly after the storm "Petra" produced heavy rainfall in the surrounding areas and caused an increase in river discharge. Vertical CTD (Conductivity, Temperature and Depth) profiles at stations located along three different transects across the fjord reveal that the strong input of fresh water from Drammen river can be tracked halfway down the Oslo fjord.

Provided is documentation of the observations and comparisons with simulations using the FjordOs CL model. The drifters were released in groups of three in two drop zones. Four drifters were released along a line in a third drop zone, and in addition two individual drifters were released as selected spots. The drifter trajectories are compared with simulated trajectories using the open source trajectory-model OpenDrift.
%\\[50mm] % Add whitespace if necessary
\\ \hline
{\bf \sffamily Keywords}                                          \\ 
  relevant, keywords, here    \\ 
\hline
\end{tabular}
}

\begin{tabular}[t]{cc}
                             &                            \\
                             &                            \\
                             &                            \\
\line(1,0){70}               & \line(1,0){70}             \\ 
Disciplinary signature       & Responsible signature      \\
%Jan Erik Haugen             & \O{}ystein Hov             \\       % Add names if needed
\hspace{75mm}                & \hspace{75mm}              \\

\end{tabular}
\end{table}

\clearpage

\thispagestyle{fancy} % footer from fancyhdr package
\headheight=15pt
\renewcommand{\headrulewidth}{0pt}

%\clearpage
\section*{\hspace{17mm}Abstract}
A field trip onboard the research vessel R/V Trygve Braarud in the Oslofjord 21 – 23 September 2015 provided observations on hydrography and trajectories of drifters. The field trip was carried out as a part of the FjordOs project as a pilot test to examinie if more field test would benefit the validation of the new Oslofjord model, the FjordOs CL model.

The cruise was carried out shortly after the storm "Petra" produced heavy rainfall in the surrounding areas and caused an increase in river discharge. Vertical CTD (Conductivity, Temperature and Depth) profiles at stations located along three different transects across the fjord reveal that the strong input of fresh water from Drammen river can be tracked halfway down the Oslo fjord.

Provided is documentation of the observations and comparisons with simulations using the FjordOs CL model. The drifters were released in groups of three in two drop zones. Four drifters were released along a line in a third drop zone, and in addition two individual drifters were released as selected spots. The drifter trajectories are compared with simulated trajectories using the open source trajectory-model OpenDrift.

\clearpage

\vfill

\fancyfoot{
% If abstract on separate page is not needed, move the following table to the page before
\begin{tabular}[b]{p{40mm}p{25mm}p{25mm}p{25mm}p{25mm}}
 \begin{minipage}[l]{40mm} \tiny \color{METblue} {\bf Norwegian Meteorological Institute}\\ Org.no 971274042\\ post@met.no\\ www.met.no / www.yr.no
 \end{minipage} & 
 \begin{minipage}[l]{25mm} \tiny \color{METblue} {\bf Oslo}\\ P.O. Box 43, Blindern\\ 0313 Oslo, Norway\\ T. +47 22 96 30 00
 \end{minipage} &
 \begin{minipage}[l]{25mm} \tiny \color{METblue} {\bf Bergen}\\ All\'egaten 70\\ 5007 Bergen, Norway\\ T. +47 55 23 66 00
 \end{minipage} & 
 \begin{minipage}[l]{25mm} \tiny \color{METblue} {\bf Troms\o}\\ P.O. Box 6314, Langnes\\ 9293 Troms\o, Norway\\ T. +47 77 62 13 00
 \end{minipage} & 
 \begin{minipage}[l]{25mm} \tiny \color{METblue} 
 \end{minipage}
\end{tabular}
}

\clearpage
\tableofcontents

\clearpage
% Begin arabic page numbering (starts on 1)
%\pagenumbering{arabic}

\section{Introduction}
Provided is a documentation of a pilot scientific cruise in the Oslofjord 21-23 September 2015. The cruise was an experiment to examine if more field test would benefit the validation of the new Oslofjord model (ref. teknisk rapport). The cruise is carried out as part of the FjordOs project.


Nine drifters were used during the cruise in addition to CTD measurements (Fig.~\ref{fig:Utstyr}). %The ships log were used to estimate the current velocity by finding the difference between speed over ground and speed through water measured at 2.2 meters depth below the keel of the vessel. 

\begin{figure}[b]
\centerline{
%\includegraphics*[width=0.65\textwidth]{Figurer/Driftere_ombord}
%\includegraphics*[width=0.36\textwidth]{Figurer/CTD}
\includegraphics*[height=7.5cm]{Figurer/Driftere_ombord}
\includegraphics*[height=7.5cm]{Figurer/CTD}
}
\caption{\small
The nine drifters before they were released (left) and the CTD (right).}
\label{fig:Utstyr}
\end{figure}

\begin{figure}[tb]
\centerline{
\includegraphics*[width=\textwidth]{Figurer/Vannstand}}
\caption{\small
Observed waterlevel at Oscarsborg in September 2015. Data obtained from www.vannstand.no.}
\label{fig:Waterlevel}
\end{figure}

\begin{figure}[tb]
\centerline{
\includegraphics*[width=\textwidth]{Figurer/River}}
\caption{\small
Observed flow rate of the Drammen river measured in Mj\o ndalen September 2015. Data obtained from NVE. The data is not quality assured.}
\label{fig:River}
\end{figure}

\begin{figure}[tb]
\centerline{
\includegraphics*[width=\textwidth]{Figurer/Gullholmen_vind_sep2015}}
\caption{\small
The hourly mean (black) and maximum (red) vind velocity at Gullholmen during the time of the cruise. Data obtained from www.yr.no.}
\label{fig:Wind}
\end{figure}


\subsection{Weather conditions}

The storm "Petra" hit Buskerud, Vestfold, Telemark, and Agder from Monday 14 to Friday 18 September 2015. The storm was caused by a main low pressure in the English channel which moved northwards crossing the North Sea. The low pressure caused an increased water level in the Oslo fjord from the 17 to 18 September before normalizing around the 20 September (Fig.~\ref{fig:Waterlevel}).

Due to the heavy rainfall that followed, rivers and streams were flooded and caused an increased fresh water flow into the fjord. There are two large rivers connected with the Oslo fjord, one in Drammen and the other in Fredrikstad. The river in Drammen had an increase from around 650 m$^3$/s to around 1550 m$^3$/s from the 15 to 18 September (Fig.~\ref{fig:River}). The volume flux slowly descended before reaching its more normal state in the beginning of November. 

During the cruise the weather was sunny with light or gentle breeze (Fig.~\ref{fig:Wind}). The direction Monday afternoon was from south  and thereafter from east. 

\subsection{Ocean model}
The FjordOs-model was set up for the time of the pilot cruise. (Link til teknisk rapport for FjordOs-modellen). Current fields at 2 meters depth (\ref{fig:Current1}-\ref{fig:Current4}) reveals a strong outflow from the Drammens fjord during the cruise. 

(Noe om styrken p\aa str\o mmen i forhold til vanlig? Temperaturfelt for \aa vise at dette er ferskvann? Mer info om FjordOs-modellen eller holder det \aa vise til den tekniske rapporten?)

%The simulated water level compared to the observerd water level...

\clearpage

% ----------------------------------------------------------------------------
\section{Observations}


\subsection{Drifter trajectories}
The drifters are made from a drainpipe 1m long, and is equipped with a foam "donut" approximately 20 cm from the top. The purpose of the donut is to stabilize the drifter, and to decrease vertical motion due to small surface waves. The pipe fills with water when it is dropped into the sea, and is also ballasted for the drifter to stay vertically. The ballast is tuned so that the donut is level with the water surface.

The Arduiono\footnote{https://www.arduino.cc/}-based control unit of the drifters, located at the top of the drifter, is equipped with a GPS-reciever and a GSM-transmitter. The refresh rate of the positions sent to our database was set to 0.0167 Hz (updates once every minute). Unfortunately, some of the drifters failed\footnote{Stopped sending data to our database. This is being looked into by the developer of the drifters.} during the cruise.

The drifters were released in groups of three in two drop zones (Fig.~\ref{fig:ZoneA} and \ref{fig:ZoneB}), and four drifters along a line in a third drop zone (Fig.~\ref{fig:ZoneC}). In addition two individual drifters were released (Fig.~\ref{fig:ZoneD}). For more details see Table~\ref{tab:Drifters}.


\begin{table}[tb]
%\vspace{-1.5cm}
{\bf Table A1: Drifters.}\\
\label{tab:Drifters}
\begin{tabular}{|c|c|ccc|} 
        \hline
        Drop & Drifter & Drop pos.            & Last sent pos.          & Picked up                  \\
        ID  & No.     & and time [UTC]       & and time [UTC]          &                            \\ \hline
             &         &          \multicolumn{3}{c|}{\textit{Drop zone A - Filtvet}}                 \\ 
        A$_1$& 1       & 59.548130, 10.617290 & 59.612400, 10.633990    & 59.581355, 10.619952       \\
             &         & 12:35 21.09.15       & 01:08 22.09.15          & 15:55 22.09.15             \\ 
        A$_2$& 8       & 59.548328, 10.617360 & 59.586281, 10.616590    & Not found                  \\
             &         & 12:26 21.09.15       & 09:42 22.09.15          &                            \\
        A$_3$& 9       & 59.548351, 10.617710 & 59.527760, 10.537500    & In last sent position      \\
             &         & 12:22 21.09.15       & 16:27 22.09.15          &                            \\ \hline 
             &         &          \multicolumn{3}{c|}{\textit{Drop zone B - R\o dtangen}}             \\
        B$_1$& 3       & 59.531219, 10.405220 & 59.507019, 10.426410    & 59.494862, 10.476880       \\
             &         & 13:35 21.09.15       & 15:05 21.09.15          & 17:54 21.09.15             \\
        B$_2$& 5       & 59.524269, 10.404090 & 59.510750, 10.428240    & In last sent position     \\
             &         & 13:39 21.09.15       & 09:09 22.09.15          &                            \\
        B$_3$& 6       & 59.524261, 10.404080 & 59.507210, 10.546380    & 59.508735, 10.433803       \\
             &         & 13:39 21.09.15       & 22:37:15 21.09.15       & 09:12 22.0915              \\
        B$_4$& 10      & 59.494839, 10.477530 & 59.519531, 10.410070    & In last sent position      \\
             &         & 17:53 21.09.15       & 09:14 22.0915           &                            \\ \hline 
             &         &          \multicolumn{3}{c|}{\textit{Drop zone C - Horten-Moss}}             \\
        C$_1$& 4       & 59.430672, 10.514930 & 59.434101, 10.510170    & 59.437445, 10.507031       \\
             &         & 10:18 22.09.15       & 12:31 22.09.15          & 14:52 22.09.15             \\
        C$_2$& 5       & 59.431519, 10.525370 & 59.428638, 10.511890    & Found in Hurum             \\
             &         & 10:26 22.09.15       & 15:53 22.09.15          &                            \\
        C$_3$& 10      & 59.430859, 10.540190 & 59.439060, 10.496760    & In last sent position      \\
             &         & 10:46 22.09.15	      & 08:50 23.09.15          &                            \\
        C$_4$& 7       & 59.432251, 10.557540 & 59.428982, 10.555320    & 59.406315, 10.532240       \\
             &         & 11:01 22.09.15       & 11:24 22.09.15          & 14:30 22.09.15             \\ \hline 
             &         &          \multicolumn{3}{c|}{\textit{Single drops}}                          \\
        D$_1$& 9       & 59.528332, 10.593090 & 59.523029, 10.520360    & 59.524059 10.520618        \\
             &         & 16:46 22.09.15       & 12:05 23.09.15          & 12:40 23.09.15             \\
%        13   & 3+6+7   & 59.505730, 10.424240 & 59.512871,  10.432890   & 59.454636 10.424795        \\
%             &         & 17:56 22.09.15       & 21:27 22.09.15          &                            \\
        D$_2$& 3+6+7   & 59.505692, 10.424270 & 59.474941, 10.431430    & 59.454636 10.424795        \\
             &         & 17:56 22.09.15       & 06:00 23.09.15          &                            \\ 
\hline 
\end{tabular}
\end{table}
\clearpage


% ----------ZONE A------------------------------------------------------------
\subsubsection{Drop zone A: Filtvet}
%
\begin{figure}[ht]
\centerline{
\includegraphics*[width=0.6\textwidth]{Figurer/zoneA}}
\caption{\small
The path and velocity of the drifters droppped in zone A.% together with the SOG estimated from the ship log.
}
\label{fig:ZoneA}
\end{figure}
%
% \begin{figure}[ht]
% \centerline{
% \includegraphics*[width=0.495\textwidth]{Opendrift_simulations/dropzone_A_no1_fjordos_arome}
% \includegraphics*[width=0.495\textwidth]{Opendrift_simulations/dropzone_A_no1_norkyst_arome}
% }
% \caption{\small
% Simulated trajectories of drifter dropped in zone A using OpenDrift (as described in Section \ref{sect:trajmod}). Left panel is forced by the FjordOs-CL1 model, and the right panel is forced by the NorKyst-800m model. Both have wind drift from the atmospheric model Arome-2.5km, but no wave drift is applied. Particles in OpenDrift were released over a period of 8 hours.
% }
% \label{fig:opendriftA}
% \end{figure}
%
\begin{figure}[ht]
\centerline{
\includegraphics*[width=0.495\textwidth]{Figurer/Driftere_vann}
\includegraphics*[width=0.495\textwidth]{Figurer/Driftere_vann}
}
\caption{\small
The drifters shortly after releasing in zone A (left) and zone B (right). OBS! Bildet til venstre skal byttes ut!
}
\label{fig:DriftereVann}
\end{figure}

Three drifters were dropped in zone A approximately at 12:30 pm on 21 September 2015. The drifters were dropped within two meters of each other (Fig.~\ref{fig:DriftereVann}, left).

At the time of deployment the currents were week due to high water (Fig.~\ref{fig:Current2}). The drifters stayed in approximately the same positions during decending tides. At low waters at 07:00 pm the drifters started to float northwards and continued for the next approximately 12 hours. The drifters had a maximum speed of 0.5 m/s which is not found in the simulated current fields (Fig.~\ref{fig:Current2}-\ref{fig:Current3}). The drifters had parted and the drifter closest to the coastline, had a weaker velocity and changed direction first (Fig.~\ref{fig:ZoneA}). 

Two of the drifters were picked up the next day. One drifter was lost. 

% ----------ZONE B------------------------------------------------------------
\subsubsection{Drop zone B: R\o dtangen}
%
\begin{figure}[ht]
\centerline{
\includegraphics*[width=0.9\textwidth]{Figurer/zoneB}}
\caption{\small
The path and velocity of the drifters droppped in zone B.}
\label{fig:ZoneB}
\end{figure}
%
% \begin{figure}[ht]
% \centerline{
% \includegraphics*[width=0.495\textwidth]{Opendrift_simulations/dropzone_B_no7_fjordos_arome}
% \includegraphics*[width=0.495\textwidth]{Opendrift_simulations/dropzone_B_no7_norkyst_arome}
% }
% \caption{\small
% As for Fig.~\ref{fig:opendriftA}, but for drifter in zone B.
% }
% \label{fig:opendriftB}
% \end{figure}
%
Three drifters were dropped in zone B approximately at 01:30 pm on 21 September 2015. The drifters were dropped close to each other (Fig.~\ref{fig:DriftereVann}, right).

The ship log revealed that the drifters were deployed close to the eastern edge of the outflow of the Drammens fjord. The drifters followed the outflow from the Drammens fjord southwards before they turned east. The speed of the drifters were approximately 0.5 m/s (Fig.~\ref{fig:ZoneB}) which is approximately the same level as in the simulated current fields (Fig.~\ref{fig:Current1}). 

One of the drifters stopped sending and were substitued at approximately 06:00 pm. The drifters floated further east towards the small island Tofteholmen lying on a south-north threshold. At 10:00 pm the drifters turned westwards. The speed of the drifters increased as they approched the outlet of the Drammens fjord. This is in accordance with the simulated current fields (Fig.~\ref{fig:Current3}). At approximately 09:00 am 22 Sepmteber the drifters were picked up north of the position they were deployed.

% ----------ZONE C------------------------------------------------------------
\subsubsection{Drop zone C: Horten-Moss}
%
\begin{figure}[ht]
\centerline{
\includegraphics*[width=0.9\textwidth]{Figurer/zoneC}}
\caption{\small
The path and velocity of the drifters droppped in zone C.}
\label{fig:ZoneC}
\end{figure}
%
% \begin{figure}[ht]
% \centerline{
% \includegraphics*[width=0.495\textwidth]{Opendrift_simulations/dropzone_C_no10_fjordos_arome}
% \includegraphics*[width=0.495\textwidth]{Opendrift_simulations/dropzone_C_no10_norkyst_arome}
% }
% \caption{\small
% As for Fig.~\ref{fig:opendriftA}, but for drifter in zone C.
% }
% \label{fig:opendriftC}
% \end{figure}
%
Four drifters were released at four positions in a cross-section of the fjord between Horten and Moss on the 22 September 2015. According to the simulated current fields (Fig.~\ref{fig:Current3}) they should have floated southwards with stronger speed the further east. 

The ship log revealed a flow towards south-southwest in the whole cross section at the time of deployment. The currents were weaker towards Horten. At 11:30 a.m. the ship log revealed a weak current towards north close to Horten. All four drifters floated towards southwest, but turned towards northeast after a few hours. The drifter closest to Horten (no. 4), changed direction at approximately 11:30, the two next (no. 5 and 10) at approximately 14:40. The last (no. 7) stopped sending its position shortly after deployment.

Only drifter no. 10 continued to send its position until it was picked up. The drifter stranded in a shallow area outside of the island Karljohansvern by Horten.

One drifter were lost (no. 5), but was found in Hurum a few days later and returned to the research group.

% ----------ZONE D------------------------------------------------------------
\subsubsection{Single drops}
%
\begin{figure}[ht]
\centerline{
\includegraphics*[width=0.9\textwidth]{Figurer/zoneD}}
\caption{\small
The path and velocity of the single drifters droppped in Breiangen.}
\label{fig:ZoneD}
\end{figure}
%
% \begin{figure}[ht]
% \centerline{
% \includegraphics*[width=0.495\textwidth]{Opendrift_simulations/dropzone_D_no12_fjordos_arome}
% \includegraphics*[width=0.495\textwidth]{Opendrift_simulations/dropzone_D_no12_norkyst_arome}
% }
% \caption{\small
% As for Fig.~\ref{fig:opendriftA}, but for single drifter in drop no. 12.
% }
% \label{fig:opendriftD1}
% \end{figure}
% %
% \begin{figure}[ht]
% \centerline{
% \includegraphics*[width=0.495\textwidth]{Opendrift_simulations/dropzone_D_no13_fjordos_arome}
% \includegraphics*[width=0.495\textwidth]{Opendrift_simulations/dropzone_D_no13_norkyst_arome}
% }
% \caption{\small
% As for Fig.~\ref{fig:opendriftA}, but for single drifter in drop no. 13.
% }
% \label{fig:opendriftD2}
% \end{figure}
%
Two single drifters were released in Breiangen (Fig.~\ref{fig:ZoneD}).

One drifter (no. 9) were released in the northeastern part of Breiangen at 16:46 on 22 September 2015. It floated slowly, first to the northeast, then south, and then north before it followed an eastern current along the coast. The simulated current fields reveal a system of unorganized eddies with weak velocities in this area at that time.

One drifter consisting of three interconnected drifters (no. 3/6/7) were released south in the northwestern part of Breiangen at 17:56 on 22 September 2015. It made a anti-clockwise turn with slow velocity an accordance with the simulated current fields (Fig.~\ref{fig:Current3}). At 01:30 the drifter started flowing southwards with the outflow from the Drammens fjord. The velocity of the drifter was almost 0.4m/s east of Lang\o ya which is also in accordance with the simulated current fields (Fig.~\ref{fig:Current4}).   

% ----------------------------------------------------------------------------

%\subsection{CTD}
%The CTD measurements were performed using an ...
%
%Measurements were performed in three crossections in addtion to some extra positions which are applied on a regular basis over several years (Fig. ?).
%
%To be continued ...
%
%\begin{figure}[ht]
%\centerline{
%\includegraphics*[width=\textwidth]{Figurer/Salt_snitt}}
%\caption{\small
%Salinity in the three crossections.}
%\label{fig:CTD_Salt}
%\end{figure}

\clearpage

% ----------------------------------------------------------------------------
\subsection{Hydrography}
Episodes of intense precipitation leads to fierce water flow in the rivers, 
like it is observed in Drammen River from September 17 (Fig. 2). % need ref
Such episodes are important for the water quality in the fjord. 
The most obvious effect is that the salinity in the surface layer of the fjord 
are reduced, but the river water also carry suspended particles, organic matter, 
nutrients as well as others substances like for instance contaminants.       

It was taken vertical profiles with a CTD (Conductivity, Temperature and Depth)
from R/V Trygve Braarud 
at stations located along three different transects across the fjord. 
The CTD instrument was of the type Seabird SBE 911 plus.
The stations are listed in Table \ref{tab:CTD} and shown at the map in 
Fig. \ref{fig:CTD_Kart}. 

The profiles of salinity and temperature reveal the different water masses in 
the fjord separated by topography. Fig. \ref{fig:profiles} show profiles from 
four different stations, Fl1 inside the Dr{\o}bak Sill, Im2 in the Dr{\o}bak Sound, 
Pj2 located between Horten and Jel{\o}ya and Sm1 outside Slagen. 
The salinity at station Fl1 show that the water mass below approximately 30 m 
are different than in the rest of the fjord. 
The sill depth at Dr{\o}bak is about 20 m. 
By zooming in on the depth range 90-160 m, it can also be seen that the water 
masses below approximately 115 m in the Dr{\o}bak Sound are different than at 
stations further out in the fjord. The sill depth between the Dr{\o}bak Sond and 
Breiangen is about 115 m. 

\begin{table}[tb]
%\vspace{-1.5cm}
{\bf Table A2: Hydrography stations.}\\
\label{tab:CTD}
\begin{tabular}{|@{}c|l|l|l|l|} \hline
Station & Date and time & Latitude & Longitude & Salinity 0-5 m \\ 
code    &  (UTC)        &          &           & (psu) \\ \hline
Fl1 & Sep 21 2015 08:19 & 59.754066 & 10.574734 & 19.09\\ \hline
Im2 & Sep 21 2015 09:18 & 59.754066 & 10.628217 & 21.93\\ \hline
D-1 & Sep 21 2015 11:29 &           &           & 17.35\\ 
    & Sep 22 2015 15:27 & 59.754066 & 10.405000 & 15.62\\ \hline
Me1 & Sep 21 2015 11:58 &           &           & 14.71\\  
    & Sep 23 2015 07:57 & 59.754066 & 10.360000 & 12.72\\ \hline
Mf1 & Sep 21 2015 12:12 &           &           & 14.66\\ 
    & Sep 23 2015 08:11 & 59.754066 & 10.391667 & 15.05\\ \hline
Mg1 & Sep 21 2015 12:28 &           &           & 19.00\\
    & Sep 21 2015 15:26 &           &           & 16.15\\
    & Sep 22 2015 15:45 &           &           & 15.84\\	
    & Sep 23 2015 08:26 & 59.754066 & 10.422500 & 17.44\\ \hline
Mh1 & Sep 21 2015 12:46 &           &           & 19.31\\ 
    & Sep 23 2015 08:45 & 59.754066 & 10.460833 & 16.96\\ \hline
Mi1 & Sep 21 2015 13:02 &           &           & 20.43\\ 
    & Sep 23 2015 09:02 & 59.754066 & 10.495833 & 16.86\\ \hline
Mj1 & Sep 21 2015 13:18 &           &           & 19.86\\ 
    & Sep 23 2015 09:14 & 59.754066 & 10.518333 & 16.96\\ \hline
Mk1 & Sep 21 2015 13:32 &           &           & 21.18\\ 
    & Sep 23 2015 09:30 & 59.754066 & 10.550000 & 19.83\\ \hline
Ml1 & Sep 21 2015 13:46 &           &           & 20.11\\ 
    & Sep 23 2015 09:44 & 59.754066 & 10.581667 & 15.68\\ \hline
Mm1 & Sep 21 2015 14:02 &           &           & 19.15\\ 
    & Sep 23 2015 10:00 & 59.754066 & 10.620000 & 14.74\\ \hline
Pi1 & Sep 22 2015 08:03 & 59.754066 & 10.508333 & 17.28\\ \hline
Pj1 & Sep 22 2015 08:25 & 59.431667 & 10.525000 & 15.91\\ \hline
Pj2 & Sep 22 2015 08:37 & 59.431667 & 10.540157 & 16.14\\ \hline
Pk2 & Sep 22 2015 08:53 & 59.432419 & 10.557180 & 17.30\\ \hline
Pl1 & Sep 22 2015 09:08 & 59.431667 & 10.573333 & 19.14\\ \hline
Sk1 & Sep 22 2015 10:24 & 59.311667 & 10.540000 & 21.74\\ \hline
Sl1 & Sep 22 2015 10:37 & 59.314167 & 10.570000 & 20.40\\ \hline
Sm1 & Sep 22 2015 10:55 & 59.316667 & 10.601667 & 17.77\\ \hline
Sn1 & Sep 22 2015 11:16 & 59.318333 & 10.628333 & 17.94\\ \hline
\end{tabular}
\end{table}

\begin{figure}[ht]
\centerline{
\includegraphics*[width=\textwidth]{Figurer/Kart_Breiangen.png}}
\caption{\small
Posistion of CTD station (red dots) taken between Sep. 21st and Sep. 23rd 2015.
The colorbar indicate the water depth.}
\label{fig:CTD_Kart}
\end{figure}

\begin{figure}[ht]
\centerline{
\includegraphics*[width=\textwidth]{Figurer/Fig_profiles.png}}
\caption{\small
Profiles of salinity (left) and temperature (right) at four stations in the 
Oslofjord. The upper panels show the depth range from 0 to 165 m.
The lower panels show th edepth range from 90 to 160 m, and station Fl1 is omitted.
}
\label{fig:profiles}
\end{figure}

At the CTD instrument, other sensors can be installed. 
On board R/V Trygve Braarud it is a Seabird SBE plus 9 CTD, 
with additional sensors for measuring oxygen concentration, turbidity, 
chlorophyll fluorescence and coloured dissolved organic matter (CDOM). 
All of these, except the oxygen sensor, are optical instruments. 
The turbidity is a measure of much light is scattered in the water, 
which depends on particle concentration. The value of the unit FNU is 
approximately equal to a particle concertation of 1 mg/L, 
but the scattering can also be affected by other things than particles, 
like water bobbles. Chlorophyll fluorescence is a measure of the amount 
of fluorescent light from algae, and is a proxy for the chlorophyll content 
in the water mass. However, the relation between chlorophyll fluorescence 
and actual chlorophyll concentration can vary considerably, 
for instance with depth. 
CDOM is also called yellow substance, and is the optically measurable 
component of the dissolved organic matter in water. 
There is usually a strong negative correlation between salinity and CDOM, 
since the rivers are the major source of organic matter to the coastal zone. 

During the cruise with R/V Trygve Braarud (Sep 21th – 23rd), three transects 
across the fjord was conducted with the CTD. Nine stations (Me1-Mm1) were taken 
across Breiangen from Lang{\o}ya in the west to Jel{\o}ya in the east. 
Five stations (Pi1-Pl1) were taken between Horten and Gullholmen. 
Four stations (Sk1-Sn1) were taken between Slagen and Edge{\o}ya. 
For each of the transect salinity, temperature, chlorophyll fluorescence, 
CDOM and oxygen saturation are shown (Figs. \ref{fig.brei1} – \ref{fig.salgen}). 
Two features are striking, 
firstly the strong input of fresh water from Drammen river and secondly the algae 
bloom on the east side of the fjord.

In Breiangen the first day of the cruise the freshest water is found on the west 
side, and on the east side the salinity is relatively high and high fluorescence 
values are found at the station closest to Jel{\o}ya. 
Two days later the fresh water is more evenly distributed across Breiangen, 
except at stations Mj1 and Mk1. At these two stations the highest fluorescence 
in the surface layer values are found. It is very clear that the algae are not 
found in the freshest part of the river water, since the low salinity is less 
optimal for marine algae. The algae are nevertheless found close to the surface 
where more light is available, and where the water column are more stratified. 
It is clear that the CDOM and the salinity pattern are very similar, and that 
CDOM is a good indicator for river water. It can also be noted that high oxygen 
concentrations can be related to high concentration of algae. 

Much of the same patter as is seen in Breiangen, is found in the transect 
between Horten and Gullholmen. The water from the Drammen River is found in 
the western part, and an algae bloom is found at the station farthest to the east. 
At the transect between Slagen and Edge{\o}ya the picture is more diffuse, with both 
the freshest and the most fluorescence rich water is found to the east. 
The salinity in this area can just as well be influenced by water from Glomma, 
given the right atmospheric conditions, but we will here explain that the low 
salinity at this transect is cause by water from Drammenelva. 
First a mixing diagram can be shown, where two water properties are plotted 
along each of the axes. Often temperature and salinity is chosen, 
but here we will use CDOM and chlorophyll fluorescence. 
If the low salinity at station Sm1 and Sn1 at 1.5 m is due to river water from 
Drammenselva, this water mass are a mixture water with high CDOM and low 
fluorescence values and water with low CDOM and high fluorescence values. 
River water meets marine water with algae. 
The points D1 and D2 (Figs. \ref{fig.brei1} and \ref{fig.brei1}) are chosen to 
represent the river water, 
and the points B1 and B2 are chosen to represent the marine water. 
From the mixture diagram (Fig. \ref{fig:CFdiagram}) it can be seen that it is 
plausible that the water masses at station Sm1 and Sn1 
(called S1 and S2 in Fig. \ref{fig.salgen}) indeed are a 
product of river water and marine water with algae. 
From the model results shows that it should be expected that the river 
water from Drammenselva is found at the east side of the Slagen transect. 


\begin{figure}[ht]
\centerline{
\includegraphics*[width=\textwidth]{Figurer/Breiangen_21_09_2015_v2.png}}
\caption{\small
Salinity in the three crossections.}
\label{fig:brei1}
\end{figure}

\begin{figure}[ht]
\centerline{
\includegraphics*[width=\textwidth]{Figurer/Breiangen_23_09_2015_v2.png}}
\caption{\small
Salinity in the three crossections.}
\label{fig:brei2}
\end{figure}

\begin{figure}[ht]
\centerline{
\includegraphics*[width=\textwidth]{Figurer/Horten_22_09_2015_v2.png}}
\caption{\small
Salinity in the three crossections.}
\label{fig:horten}
\end{figure}

\begin{figure}[ht]
\centerline{
\includegraphics*[width=\textwidth]{Figurer/Slagen_22_09_2015_v2.png}}
\caption{\small
Salinity in the three crossections.}
\label{fig:slagen}
\end{figure}

\begin{figure}[ht]
\centerline{
\includegraphics*[width=\textwidth]{Figurer/CF_diagram.png}}
\caption{\small
Salinity in the three crossections.}
\label{fig:CFdiagram}
\end{figure}

\clearpage

\section{Simulations}

\subsection{Trajectories}
\label{sect:trajmod}
To simulate the trajectories of the drifters, we have applied the open source trajectory-model OpenDrift. This is a trajectory model under development at MET Norway, and is described by its developers as "a software for modeling the trajectories and fate of objects or substances drifting in the ocean, or even in the atmosphere". It is distributed under a GPL v2.0 license, and is available online\footnote{https://github.com/knutfrode/opendrift}.


\begin{figure}[ht]
\centerline{
\includegraphics*[width=0.495\textwidth]{Opendrift_simulations/LTR3/tokt_drifters_winddrift_0p0_radius_0_num_6_plusminus_2p5h_crop}
\includegraphics*[width=0.495\textwidth]{Opendrift_simulations/LTR3/tokt_drifters_winddrift_0p0_radius_0_num_6_plusminus_2p5h_norkyst_crop}
}
\caption{\small
Caption here.
}
\label{fig:opendriftD2}
\end{figure}

\begin{figure}[ht]
\centerline{
\includegraphics*[width=0.495\textwidth]{Opendrift_simulations/LTR3/tokt_drifters_winddrift_0p1_radius_0_num_6_plusminus_2p5h_crop}
\includegraphics*[width=0.495\textwidth]{Opendrift_simulations/LTR3/tokt_drifters_winddrift_0p1_radius_0_num_6_plusminus_2p5h_norkyst_crop}
}
\caption{\small
Caption here.
}
\label{fig:opendriftD2}
\end{figure}

\begin{figure}[ht]
\centerline{
\includegraphics*[width=0.495\textwidth]{Opendrift_simulations/LTR3/tokt_drifters_winddrift_0p2_radius_0_num_6_plusminus_2p5h_crop}
\includegraphics*[width=0.495\textwidth]{Opendrift_simulations/LTR3/tokt_drifters_winddrift_0p2_radius_0_num_6_plusminus_2p5h_norkyst_crop}
}
\caption{\small
Caption here.
}
\label{fig:opendriftD2}
\end{figure}

\begin{figure}[ht]
\centerline{
\includegraphics*[width=0.495\textwidth]{Opendrift_simulations/LTR3/tokt_drifters_winddrift_0p5_radius_0_num_6_plusminus_2p5h_crop}
\includegraphics*[width=0.495\textwidth]{Opendrift_simulations/LTR3/tokt_drifters_winddrift_0p5_radius_0_num_6_plusminus_2p5h_norkyst_crop}
}
\caption{\small
Caption here.
}
\label{fig:opendriftD2}
\end{figure}


\subsection{Hydrography}

To be continued... (resultater fra FjordOs modellen)

\clearpage

% ----------------------------------------------------------------------------
\section{Discussion \& Conclusions}

To be continued...


\clearpage
\section*{\hspace{17mm}Acknowledgements}
Thanks to the crew at the research vessel R/V Trygve Braarud.

\clearpage
\section*{\hspace{17mm}Appendix}

\begin{figure}[h]
\centerline{
\includegraphics*[trim=2.0cm 3cm 6.0cm 3.5cm,clip=true,height=7cm]{Python/stromfelt_10}
\includegraphics*[trim=3.7cm 3cm 1.3cm 3.5cm,clip=true,height=7cm]{Python/stromfelt_12}
}
\centerline{
\includegraphics*[trim=2.0cm 3cm 6.0cm 3.5cm,clip=true,height=7cm]{Python/stromfelt_14}
\includegraphics*[trim=3.7cm 3cm 1.3cm 3.5cm,clip=true,height=7cm]{Python/stromfelt_16}
}
\centerline{
\includegraphics*[trim=2.0cm 3cm 6.0cm 3.5cm,clip=true,height=7cm]{Python/stromfelt_18}
\includegraphics*[trim=3.7cm 3cm 1.3cm 3.5cm,clip=true,height=7cm]{Python/stromfelt_20}
}
\caption{\small
Simulated current fields at 2 meters depth every other hour during the cruise.}
\label{fig:Current1}
\end{figure}

\begin{figure}[h]
\centerline{
\includegraphics*[trim=2.0cm 3cm 6.0cm 3.5cm,clip=true,height=7cm]{Python/stromfelt_22}
\includegraphics*[trim=3.7cm 3cm 1.3cm 3.5cm,clip=true,height=7cm]{Python/stromfelt_24}
}
\centerline{
\includegraphics*[trim=2.0cm 3cm 6.0cm 3.5cm,clip=true,height=7cm]{Python/stromfelt_26}
\includegraphics*[trim=3.7cm 3cm 1.3cm 3.5cm,clip=true,height=7cm]{Python/stromfelt_28}
}
\centerline{
\includegraphics*[trim=2.0cm 3cm 6.0cm 3.5cm,clip=true,height=7cm]{Python/stromfelt_30}
\includegraphics*[trim=3.7cm 3cm 1.3cm 3.5cm,clip=true,height=7cm]{Python/stromfelt_32}
}
\caption{\small 
Simulated current fields at 2 meters depth every other hour during the cruise.}
\label{fig:Current2}
\end{figure}

\begin{figure}[h]
\centerline{
\includegraphics*[trim=2.0cm 3cm 6.0cm 3.5cm,clip=true,height=7cm]{Python/stromfelt_34}
\includegraphics*[trim=3.7cm 3cm 1.3cm 3.5cm,clip=true,height=7cm]{Python/stromfelt_36}
}
\centerline{
\includegraphics*[trim=2.0cm 3cm 6.0cm 3.5cm,clip=true,height=7cm]{Python/stromfelt_38}
\includegraphics*[trim=3.7cm 3cm 1.3cm 3.5cm,clip=true,height=7cm]{Python/stromfelt_40}
}
\centerline{
\includegraphics*[trim=2.0cm 3cm 6.0cm 3.5cm,clip=true,height=7cm]{Python/stromfelt_42}
\includegraphics*[trim=3.7cm 3cm 1.3cm 3.5cm,clip=true,height=7cm]{Python/stromfelt_44}
}
\caption{\small
Simulated current fields at 2 meters depth every other hour during the cruise.}
\label{fig:Current3}
\end{figure}

\begin{figure}[h]
\centerline{
\includegraphics*[trim=2.0cm 3cm 6.0cm 3.5cm,clip=true,height=7cm]{Python/stromfelt_46}
\includegraphics*[trim=3.7cm 3cm 1.3cm 3.5cm,clip=true,height=7cm]{Python/stromfelt_48}
}
\centerline{
\includegraphics*[trim=2.0cm 3cm 6.0cm 3.5cm,clip=true,height=7cm]{Python/stromfelt_50}
\includegraphics*[trim=3.7cm 3cm 1.3cm 3.5cm,clip=true,height=7cm]{Python/stromfelt_52}
}
\centerline{
\includegraphics*[trim=2.0cm 3cm 6.0cm 3.5cm,clip=true,height=7cm]{Python/stromfelt_54}
\includegraphics*[trim=3.7cm 3cm 1.3cm 3.5cm,clip=true,height=7cm]{Python/stromfelt_56}
}
\caption{\small
Simulated current fields at 2 meters depth every other hour during the cruise.}
\label{fig:Current4}
\end{figure}



%\begin{table}[h]
%\vspace{-1.5cm}
%{\bf Table A1: ESD statistics.}\\
%\label{tab:1}
%\begin{tabular}{llll}
%\small OSLO - BLINDERN 18700 & 
%\small $T_m$ $R^2$= 71 -- 86 $T_x$ $R^2$= 69 -- 82 $T_n$ $R^2$= 49 -- 76 \\
%\small $\Delta T_m$ $q_{0.05}$= 3.21 $q_{0.50}$= 1.33 $q_{0.95}$= 1.75 \\  
%\small $\Delta T_x$ $q_{0.05}$= 3.09 $q_{0.50}$= 1.38 $q_{0.95}$= 2.16 \\  
%\small $\Delta T_n$ $q_{0.05}$= 3.15 $q_{0.50}$= 1.12 $q_{0.95}$= 1.01 \\
%\end{tabular}
%\end{table}

\clearpage
\pagebreak

%\bibliography{refs}
%\begin{thebibliography}{1}
%\end{thebibliography}

\clearpage
\pagebreak
 

\end{document}


